\documentclass[aspectratio=169]{beamer}
\usepackage{ulem}
\usepackage{tikz}
\usepackage{booktabs}
\usepackage{graphicx,threeparttable,caption}
\usetikzlibrary{shapes,snakes}
\usepackage[beamer,customcolors]{hf-tikz}
\usepackage{nicematrix}
\usepackage{xcolor}
\usepackage{makecell}
\usepackage{array}
\usepackage{csquotes}
\usepackage{csquotes}
\usepackage{minted}
\captionsetup{labelformat=empty,labelsep=none}

\graphicspath{ {./png/} }

\usetikzlibrary{
    arrows,
    arrows.meta,
    shapes,
    positioning,
    shadows,
    trees,
    calc
}

\tikzset{%
    >={Latex[width=2mm,length=2mm]},
    % Specifications for style of nodes:
    plain/.style = {},
    base/.style = {
        plain,
        rectangle, rounded corners, draw=black,
        minimum width=1cm, minimum height=1cm,
        text centered, font=\sffamily\tiny\bfseries,
        fill=white, align=center
    },
    app/.style = {base, ellipse},
    data/.style = {base, fill=gray!30},
    action/.style = {base, circle, fill=red!30},
    note/.style = {app, fill=yellow},
    hl/.style={
    set fill color=red!80!black!40,
    set border color=red!80!black
    }
}


\AtBeginSection[]{
  \begin{frame}
  \vfill
  \centering
  \begin{beamercolorbox}[sep=8pt,center,shadow=true,rounded=true]{title}
    \usebeamerfont{title}\insertsectionhead\par%
  \end{beamercolorbox}
  \vfill
  \end{frame}
}
\setbeamercolor{alerted text}{fg=red}
%\usecolortheme[orchid]{structure}
\usetheme[hideothersubsections]{PaloAlto}
\makeatletter
\patchcmd{\csq@bquote@i}{{#6}}{{\emph{#6}}}{}{}
\makeatother
%\usecolortheme{orchid}
%\usefonttheme{professionalfonts}
\newcommand{\soutthick}[1]{%
   \textcolor{red}{
   \renewcommand{\ULthickness}{1pt}%
      \sout{#1}%
   \renewcommand{\ULthickness}{.4pt}% Resetting to ulem default
   }
}
\newcommand{\centered}[1]{\begin{tabular}{l} #1 \end{tabular}}
\setbeamertemplate{section in toc}[square]
\setbeamertemplate{subsection in toc}[square]
\setbeamertemplate{section in sidebar}[shaded]
\setbeamertemplate{items}[square]
\setbeamercovered{transparent} 

\title[]{Introduction to Computational Social Science}
\subtitle{Networks -- Discussion}
\author[]{Mikołaj Biesaga\\ \small{\color{blue}{\href{mailto:m.biesaga@uw.edu.pl}{m.biesaga@uw.edu.pl}}}}
\institute{\includegraphics[width = 4 cm]{uw.png}}
\date{\today}
\begin{document}
\begin{frame}
   \titlepage
\end{frame}

\begin{frame}
   \frametitle{The power of weak ties}
   \only<1>{
      \begin{figure}
         \centering
         \includegraphics[width = .5\textwidth]{network.png}
         \caption{from \href{https://interactioninstitute.org/network-analysis-for-change-collaborations-clusters-champions-and-coach-weavers/}{\textcolor{blue}{Ogden, 2020}}}
      \end{figure}
   }
   \only<2>{
      \framesubtitle{Granovetter, 1973}
      \begin{block}{}
         Granovetter (1973) identified weak ties as a key source of “diffusion
         of influence and information, mobility opportunity, and community
         organization.”
      \end{block}
   }
   \only<3,5>{
      \framesubtitle{Rajkumar et al., 2022}
      \begin{itemize}
         \item A five-year experiment involving LinkedIn’s “People You May Know”
         (PYMK) algorithm, which suggests new connections to site users. 
         \item A weak tie was defined two fold:
         \begin{itemize}
            \item message intensity
            \item number of mutual friends
         \end{itemize}
         \item The experiment involved around \alert{20 million} LinkedIn users, who
         over the five years ended up creating about \alert{2 billion new connections}
         on the site, recorded over \alert{70 million job applications}, and wound up
         accepting \alert{600,000 new jobs} identified through the site.
      \end{itemize}

   }
   \only<4>{
      \begin{figure}
         \centering
         \includegraphics[width = .8\textwidth]{weak_ties.png}
         \caption{Rajkumar et al., 2022}
      \end{figure}
   }
   \only<6>{
      \begin{figure}
         \begin{minipage}{.49\textwidth}
            \includegraphics[width = \textwidth]{interaction_intensity.png}
         \end{minipage}
         \begin{minipage}{.49\textwidth}
            \includegraphics[width = \textwidth]{number_friends.png}
         \end{minipage}
         \caption{Rajkumar et al., 2022}
      \end{figure}

   }
\end{frame}

\begin{frame}
   \frametitle{The strength of long-range ties}
   \framesubtitle{Park et al., 2018}
   \only<1>{
      \begin{itemize}
         \item Information from small circle of friends is more likely to be redundant than information acquired from an acquaintance in a distant region of a social network (Granovetter, 1973).
         \item There is a trade-off between the diversity of information acquired through weak bridging ties and the volume of information, or bandwidth, acquired through strong, structurally embedded ties (Park et al., 2018).
         \item Tie strength decreases as social ties become less embedded, i.e., as they connect individuals with fewer network “neighbors” in common (Onnela et al., 2007).
      \end{itemize}
      \begin{figure}
         \includegraphics[width = .7\textwidth]{362_1410_f1.jpeg}
      \end{figure}
   }
   \only<2>{
      \begin{block}{}
         However, the tendency for social networks to be highly clustered means that long range-ties are rarely observed in networks with no more than a few thousand nodes, such as villages, schools, and workplaces.
      \end{block}
   }
   \only<3>{
   Data collected from 11 population-scale communication networks from culturally and economically diverse populations spanning four continents:
      \begin{itemize}
         \item  three independent nationwide \alert{phone networks} (in Afghanistan, Rwanda, and a large European country)
         \item \alert{56 million Twitter users} in eight countries with relatively high Twitter penetration (United States, United Kingdom, France, Netherlands, Japan, South Korea, Singapore, and Turkey).
      \end{itemize}
   }
   \only<4>{
      \begin{figure}
         \includegraphics[width = .7\textwidth]{362_1410_f2.jpeg}
         \caption{Results are shown for eight Twitter networks (A) and three phone networks (B). Tie strength is measured as the log of the frequency of bidirected \@mentions (A) and the log of total bidirected call volume in seconds (B).}
      \end{figure}
   }
   \only<5>{
      \begin{minipage}{.48\textwidth}
         \begin{figure}
            \centering
            \includegraphics[width = \textwidth]{362_1410_f3.jpeg}
            %\caption{Singapore’s Twitter network, in which a tie is composed of one or more reciprocated @mentions. The wormholes, defined here as ties above range six and above median tie strength make up 0.46\% of all ties.}
         \end{figure}
      \end{minipage}
      \hfill
      \begin{minipage}{.48\textwidth}  
         < Singapore’s Twitter network, in which a tie is composed of one or more reciprocated @mentions. The wormholes, defined here as ties above range six and above median tie strength make up 0.46\% of all ties.
         \begin{block}{}
            These high-bandwidth long-range ties as network \alert{wormholes}, though relatively rare, can provide high-bandwidth shortcuts across vast reaches of network space.
         \end{block}
      \end{minipage}
   }
   \only<6>{
      The follow-up questions on the diffusion of information are:
      \begin{enumerate}
         \item From whom are we more likely to receive new information?
         \item Who is more likely to receive new information?
      \end{enumerate}
   }
   \only<7>{
      \begin{minipage}{.48\textwidth}
         \begin{figure}
            \centering
            \includegraphics[width = \textwidth]{362_1410_f4.jpeg}
         \end{figure}
      \end{minipage}
      \hfill
      \begin{minipage}{.48\textwidth}  
         \begin{itemize}
         \item Within-individual relationship where the z-score is calculated by standardizing each tie with the individual’s average and standard deviation of tie strength. 
         \item Between-individual relationship where the z-score is calculated by standardizing each individual’s average tie strength with the grand mean and standard deviation of the entire network. 
         \end{itemize}
      \end{minipage}
   }
   \only<8,10,13>{
      \begin{enumerate}
         \item<8,10, 13> In all instances, they observed that \alert<8>{tie strength eventually increased with range}.
         \item<10,13> The content of the messages exchanged over strong, long-range Twitter ties displayed no single characteristic pattern. However, temporal analysis suggested that \alert<10>{network wormholes} were more likely to be \alert<10>{interpersonal social relationships} rather than instrumental or work-related.
         \item<13> Physical and network distances were conceptually and empirically distinct dimensions. However, results were consistent with previous findings that tie \alert<13>{strength generally decreases with spatial distance, but the pattern was the opposite for network distance.}
      \end{enumerate}
   }
   \only<9>{
      \begin{figure}
         \includegraphics[width = .85\textwidth]{conversation_length.png}
      \end{figure}
   }
   \only<11>{
      \begin{figure}
         \includegraphics[width = .85\textwidth]{affect.png}
      \end{figure}
   }
   \only<12>{
      \begin{figure}
         \includegraphics[width = .85\textwidth]{job.png}
      \end{figure}
   }
\end{frame}

\begin{frame}
   \frametitle{Discussion}
   \begin{enumerate}
      \item To study what phenomena the topological approach to social networks is best suited? To study what phenomena the alogithmic approach to social networks is best suited?
      \item Write out on a piece of paper people you would contact to seek support? Now write out on a piece of paper people you would contant to
      seek legal advice? Compare the two lists. How many people are on both lists? Why?
      \item Can weak ties be harmful? In what situations?
      \item Imagine you would like to study the diffusion of information at the Department of Psychology. How would you design the study? What data would you collect? How would you analyze it?
   \end{enumerate}
\end{frame}

\begin{frame}
   \frametitle{Bibliography}
   \footnotesize
   \begin{enumerate}
      \item Granovetter, M. S. (1973). The Strength of Weak Ties. American Journal of Sociology, 78(6), 1360–1380.
      \item Park, P. S., Blumenstock, J. E., \& Macy, M. W. (2018). The strength of long-range ties in population-scale social networks. Science, 362, 1410-1413. \href{https://doi.org/10.1126/science.aau9735}{\textcolor{blue}{https://doi.org/10.1126/science.aau9735}}
      \item Rajkumar, K., Saint-Jacques, G., Bojinov, I., Brynjolfsson, E., \& Aral, S. (2022). A causal test of the strength of weak ties. Science, 377(6612), 1304–1310. \href{https://doi.org/10.1126/science.abl4476}{\textcolor{blue}{https://doi.org/10.1126/science.abl4476}}
      \item Onnela, J. P., Saramäki, J., Hyvönen, J., Szabó, G., Lazer, D., Kaski, K., ... \& Barabási, A. L. (2007). Structure and tie strengths in mobile communication networks. Proceedings of the national academy of sciences, 104(18), 7332-7336.
   \end{enumerate}
\end{frame}

\end{document}