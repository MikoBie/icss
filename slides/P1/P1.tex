\documentclass{beamer}
\usepackage{ulem}
\usepackage{tikz}
\usepackage{booktabs}
 \usepackage{graphicx,threeparttable,caption}
\usetikzlibrary{shapes,snakes}
\usepackage[beamer,customcolors]{hf-tikz}
\usepackage{nicematrix}
\usepackage{xcolor}
\usepackage{makecell}
\usepackage{array}
\usepackage{csquotes}
\usepackage{csquotes}
\usepackage{minted}
\captionsetup{labelformat=empty,labelsep=none}

\graphicspath{ {./png/} }
\tikzset{hl/.style={
    set fill color=red!80!black!40,
    set border color=red!80!black,
  },
}
\AtBeginSection[]{
  \begin{frame}
  \vfill
  \centering
  \begin{beamercolorbox}[sep=8pt,center,shadow=true,rounded=true]{title}
    \usebeamerfont{title}\insertsectionhead\par%
  \end{beamercolorbox}
  \vfill
  \end{frame}
}
%\usecolortheme[orchid]{structure}
\usetheme[hideothersubsections]{PaloAlto}
\makeatletter
\patchcmd{\csq@bquote@i}{{#6}}{{\emph{#6}}}{}{}
\makeatother
%\usecolortheme{orchid}
%\usefonttheme{professionalfonts}
\newcommand{\soutthick}[1]{%
   \textcolor{red}{
   \renewcommand{\ULthickness}{1pt}%
      \sout{#1}%
   \renewcommand{\ULthickness}{.4pt}% Resetting to ulem default
   }
}
\newcommand{\centered}[1]{\begin{tabular}{l} #1 \end{tabular}}
\setbeamertemplate{section in toc}[square]
\setbeamertemplate{subsection in toc}[square]
\setbeamertemplate{secion in sidebar}[shaded]
\setbeamertemplate{items}[square]
\setbeamercovered{transparent} 

\title[]{Introduction to Computational Social Science}
\subtitle{Introduction}
\author[]{Mikołaj Biesaga\\ \small{\color{blue}{\href{mailto:m.biesaga@uw.edu.pl}{m.biesaga@uw.edu.pl}}}}
\institute{\includegraphics[width = 4 cm]{uw.png}}
\date{\today}
\begin{document}
\begin{frame}
   \titlepage
\end{frame}

\begin{frame}
    \frametitle{Mikołaj Biesaga}
    \only<1>{
        \framesubtitle{Best Book Ever}
        \begin{center}
            \includegraphics[width = .4\textwidth]{png/gww.jpg}
        \end{center}
    }
    \only<2>{
        \framesubtitle{Current Book}
        \begin{center}
            \includegraphics[width = .4\textwidth]{the_road.jpg}
        \end{center}
    }
    \only<3>{
        \framesubtitle{Next Graphic Novel}
        \begin{center}
            \includegraphics[width = .5\textwidth]{the_road_comics.jpg}
        \end{center}
    }
    \only<4>{
        \framesubtitle{Next Series}
        \begin{center}
            \includegraphics[width = .45\textwidth]{3_body_problem.jpg}
        \end{center}
    }
\end{frame}

\section{Rules of Engagement}

\begin{frame}
    \frametitle{Office Hours, Emails, Presentations, etc.}
    \begin{description}[Google Classroom:]
        \item [Office Hours:] write me an email before coming
        \item [Emails:] the official info will go through emails
        \item [GitHub:] scripts (notebooks) will be posted on GitHub
        \item [Google Classroom:] materials and presentations will be posted on
        Google Classroom
    \end{description}
    \alert{I will try to answer your inquiries as soon as possible but do not
    count on an immediate response, especially right before the deadlines.}
\end{frame}
\begin{frame}
    \frametitle{Emails}
        \includegraphics[width = \textwidth]{emails.png}
\end{frame}
\begin{frame}
    \frametitle{Workflow}
    \begin{enumerate}
        \item Presentation of the basic concepts in the classroom
        \item Exercises in the classroom
        \item Homework assignments requiring modifying the work done in the classroom
        \item Final research project
    \end{enumerate}
\end{frame}
\begin{frame}
    \frametitle{Assessment Criteria}
    \begin{itemize}
        \item The final grade will be determined based on \alert{6 homework
        assignments} and 
        \item The first 4 homework (after classes: 2, 3, 4, and 5) will be scored from 0 to 10 points and the
        other tow (after classes: 12 and 15) from 0 to 15 points.
        \item The research project will require writing a research proposal that uses the tools and methods discussed during the class. It will be scored from 0 to 85 points.
        \item Grading criteria:
    \begin{description}
        \item [5\phantom{+} --] => 140 points
        \item [4+ --] 132 - 139 points
        \item [4\phantom{+} --] 117 - 131 points
        \item [3+ --] 109 - 116 points
        \item [3\phantom{+} --] 93 - 108 points
        \item [2\phantom{+} --] =< 92 points
    \end{description}
\end{itemize}
\end{frame}
\begin{frame}
    \frametitle{Attendance}
    \only<1>{
        \begin{itemize}
            \item Attendance is advised
            \item You are allowed to miss up to \alert{2 classes in case of a formal
            excuse}
            \item Additionally, you are allowed to miss up to \alert{2 classes in the case of a formal excuse}
            \item An absence does not exempt from doing homework assignments
        \end{itemize}
    }
    \only<2>{
        \begin{center}
            \includegraphics[width = .9\textwidth]{png/pretty_please.jpg}
        \end{center}
    }
\end{frame}

\begin{frame}
    \frametitle{Additional Resources}
    \only<1>{
        \begin{center}
            \includegraphics<1>[width = .5\textwidth]{copy_paste.png}
        \end{center}
    }
    \only<2>{
        \begin{center}
            \includegraphics<2>[width = .6\textwidth]{stack_overflow.png}
        \end{center}
    }
    \only<3>{
        \begin{itemize}
            \item \textcolor{blue}{\href{https://stackoverflow.com}{www.stackoverflow.com}}
            \item \textcolor{blue}{\href{https://www.learnpython.org}{www.learnpython.org}}
            \item \textcolor{blue}{\href{https://www.edx.org/course/introduction-to-computer-science-and-programming-7?index=product&queryID=51a933646c74acd8353b1d8c34fa59d6&position=2}{Introduction to Computer Science and Programming Using Python}}
            \item Introduction to Computation and Programming Using Python by John V. Guttag
        \end{itemize}
    }
    \only<4>{
        \begin{center}
            \includegraphics<4>[width=\textwidth]{chatgpt_stackoverflow.png}
        \end{center}
    }
    \only<5>{
        \begin{center}
            \includegraphics<5>[width=\textwidth]{chatgpt5.png}
        \end{center}
    }
    \only<6>{
        \begin{center}
            \includegraphics<6>[width=.5\textwidth]{chatgpt_meme.jpg}
        \end{center}
    }
    \only<7>{
        \begin{center}
            \includegraphics<7>[width=.9\textwidth]{chatgpt_meme2.jpg}
        \end{center}
    }
    \only<8>{
        \begin{center}
            \includegraphics<8>[width=.9\textwidth]{chatgpt_meme3.jpg}
        \end{center}
    }
\end{frame}

\section{CSS}

\begin{frame}
    \frametitle{What is Computational Social Science?}
    \begin{definition}{}
        In the most general sense \emph{Computational Social Science} is a data-driven approach that uses computational methods in studying social phenomena.
    \end{definition}
    \begin{definition}{}
        \emph{Data Science} on the other hand is a broader term than Computational Social Science. It describes the theory and practice of extracting knowledge and insight from data.
    \end{definition}


\end{frame}

\section{Python}

\begin{frame}
    \only<+>{
        \centering
        \includegraphics[width = \framewidth]{png/python_logo.png}
    }
    \only<2,3>{
        \frametitle{What is Python?}
        \only<2>{
            \begin{definition}
                \emph{Pythons} are a family of nonvenomous snakes found in Africa, Asia, and Australia. Among its members are some of the largest snakes in the world. Ten genera and 42 species are currently recognized.
            \end{definition}
        }
        \only<3>{
            \begin{definition}
                \emph{Python} is a programming language and the Python interpreter is a piece of software that reads the source code and performs its instructions. There are two generations of Python: Python 2.x and Python 3.x. \alert{We will use Python 3.x only.}
            \end{definition}
        }
    }
    \only<4>{
        \frametitle{Python}
        \includegraphics[width = \framewidth]{png/python_idle.png}
    }
\end{frame}

\begin{frame}
    \frametitle{How Social Scientists can use Python?}
    \only<1,3,5,7,9,11>{
        \begin{itemize}
            \item<1> extraction of unstructured data from external digital (i.e. web-based) sources
            \item<3> analysis of textual data (natural language processing -- NLP)
            \item<5> designing experiments
            \item<7> working with big datasets 
            \item<9> network and relational data analysis
            \item<11> computer simulations
        \end{itemize}
    }
    \only<2>{
        \includegraphics[width = \textwidth]{png/melody.png}
    }
    \only<4>{
        \includegraphics[width = \textwidth]{png/nlp.png}
    }
    \only<6>{
        \includegraphics[width = \textwidth]{png/designing.png}
    }
    \only<8>{
        \includegraphics[width = \textwidth]{png/big_data.png}
    }
    \only<10>{
        \includegraphics[width = \textwidth]{png/networks.png}
    }
    \only<12>{
        \includegraphics[width = \textwidth]{png/simulations.png}
    }

\end{frame}

\section{Objectives}

\begin{frame}
    \frametitle{Objective of the course}
    \begin{itemize}
        \item<1> present basic programming concepts in Python
        \item<1> teach you how to solve basic computing problems with the use of Python
        \item<1> teach you the importance of writing readable and reproducible code
        \item<1> present basic concepts of computational social science
        \item<1> present the advantages, challenges, and limitations of computational methods in social sciences
        \item<1> teach you how to plan research using computational methods (especially webscraping, web API data extraction, and natural language processing)
        
    \end{itemize}
    \only<2>{
        \begin{tikzpicture}[overlay]
            \node[anchor = base, text=red, text width = 13cm, align = center] at ([xshift=-1cm,yshift=-1cm]current page.center){\LARGE You will not be a Computer Scientist\\ after the course!};
        \end{tikzpicture}
    }
\end{frame}

\section{Tools}

\begin{frame}
    \only<+>{
        \frametitle{Tools of the trade}
        \centering
        \includegraphics[width = .6\framewidth]{png/jupyter.png}
    }
    \only<+>{
        \frametitle{What is Jupyter Notebook?}
        \begin{definition}
            \emph{Jupyter Notebooks} are documents produced by the Jupyter Notebook App which contain both computer code (e.g. python) and rich text elements (paragraph, equations, figures, links, etc.). Notebook documents are both human-readable documents containing the analysis description and the results (figures, tables, etc.) as well as executable documents that can be run to perform data analysis.
        \end{definition}
    }
    \only<+>{
        \frametitle{Jupyter Notebook}
        \includegraphics[width = \framewidth]{png/jupyter_notebook.png}
    }
    \only<+>{
        \frametitle{What is Jupyter Notebook App?}
        \begin{definition}
            \emph{The Jupyter Notebook App} is an application that allows editing and running code via a web browser. The app can be run on a local desktop requiring no internet access or can be installed on a remote server and accessed through the internet.
        \end{definition}
    }
\end{frame}

\begin{frame}[fragile]
    \only<+>{
        \begin{center}
            \includegraphics[scale = .7]{png/colab.png}
        \end{center}
    }
    \only<+>{
        \frametitle{What is Colaboratory?}
        \begin{definition}
            \emph{Colaboratory} is a free Jupyter Notebook environment that requires no setup and runs entirely in the cloud. In other words, it is almost the same as Jupyter Notebook App but it runs in Google Cloud, so if you have a google account you have access to Colaboratory.
        \end{definition}
    }
    \only<+>{
        \frametitle{Colaboratory}
        \includegraphics[width = \framewidth]{png/colaboratory.png}
    }
    \only<+>{
        \frametitle{How to open Jupyter Notebook in Colaboratory?}
        If you have a Google account and all of you have you just need to follow these four easy steps to open a notebook we prepared for today:
        \begin{enumerate}
            \item Visit \textcolor{blue}{\href{https://colab.research.google.com/notebooks/welcome.ipynb}{www.colab.research.google.com}}
            \item Press File and choose Open notebook\dots
            \item Choose GitHub and type \mintinline{bash}{MikoBie}
            \item Select \mintinline{bash}{icss}
            \item Click on \mintinline{bash}{notebooks/N1.ipynb}
        \end{enumerate}
    }
    \only<+>{
        \includegraphics[width = \framewidth]{png/colab_notebook.png}
    }
\end{frame}


\end{document}