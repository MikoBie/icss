\documentclass{beamer}
\usepackage{ulem}
\usepackage{tikz}
\usepackage{booktabs}
 \usepackage{graphicx,threeparttable,caption}
\usetikzlibrary{shapes,snakes}
\usepackage[beamer,customcolors]{hf-tikz}
\usepackage{nicematrix}
\usepackage{xcolor}
\usepackage{makecell}
\usepackage{array}
\usepackage{csquotes}
\usepackage{csquotes}
\usepackage{minted}
\captionsetup{labelformat=empty,labelsep=none}

\graphicspath{ {./png/} }

\usetikzlibrary{
    arrows,
    arrows.meta,
    shapes,
    positioning,
    shadows,
    trees,
    calc
}

\tikzset{%
    >={Latex[width=2mm,length=2mm]},
    % Specifications for style of nodes:
    plain/.style = {},
    base/.style = {
        plain,
        rectangle, rounded corners, draw=black,
        minimum width=1cm, minimum height=1cm,
        text centered, font=\sffamily\tiny\bfseries,
        fill=white, align=center
    },
    app/.style = {base, ellipse},
    data/.style = {base, fill=gray!30},
    action/.style = {base, circle, fill=red!30},
    note/.style = {app, fill=yellow},
    hl/.style={
    set fill color=red!80!black!40,
    set border color=red!80!black
    }
}


\AtBeginSection[]{
  \begin{frame}
  \vfill
  \centering
  \begin{beamercolorbox}[sep=8pt,center,shadow=true,rounded=true]{title}
    \usebeamerfont{title}\insertsectionhead\par%
  \end{beamercolorbox}
  \vfill
  \end{frame}
}
\setbeamercolor{alerted text}{fg=red}
%\usecolortheme[orchid]{structure}
\usetheme[hideothersubsections]{PaloAlto}
\makeatletter
\patchcmd{\csq@bquote@i}{{#6}}{{\emph{#6}}}{}{}
\makeatother
%\usecolortheme{orchid}
%\usefonttheme{professionalfonts}
\newcommand{\soutthick}[1]{%
   \textcolor{red}{
   \renewcommand{\ULthickness}{1pt}%
      \sout{#1}%
   \renewcommand{\ULthickness}{.4pt}% Resetting to ulem default
   }
}
\newcommand{\centered}[1]{\begin{tabular}{l} #1 \end{tabular}}
\setbeamertemplate{section in toc}[square]
\setbeamertemplate{subsection in toc}[square]
\setbeamertemplate{section in sidebar}[shaded]
\setbeamertemplate{items}[square]
\setbeamercovered{transparent} 

\title[]{Introduction to Computational Social Science}
\subtitle{Presentations}
\author[]{Mikołaj Biesaga\\ \small{\color{blue}{\href{mailto:m.biesaga@uw.edu.pl}{m.biesaga@uw.edu.pl}}}}
\institute{\includegraphics[width = 4 cm]{uw.png}}
\date{\today}
\begin{document}
\begin{frame}
   \titlepage
\end{frame}

\section[Presentations]{Presentations}
\begin{frame}{Structure of the presentation}
    \begin{itemize}
     \item \textbf{Introduction and problem statement.} This section should
     explain what is the project about in general and why the given problem
     matters. The significance of the problem may be justified both in terms of
     its theoretical/scientific or societal/business relevance.
     \item \textbf{Research question.} What is the specific research question
     the paper tries to answer? It can be either formulated as a
     strict hypothesis or as an exploratory question.
     \item \textbf{Research methods.} This section should discuss, at a general
     level, what data was collected and how.     
    \end{itemize}
\end{frame}
    
\begin{frame}{Structure of the presentation}
    \begin{itemize}
     \item \textbf{Analytic methods.} What analytic methods were applied to
     answer the research question? This may include some methods
     that we discussed in the class (i.e. sentiment analysis or some other
     natural language processing methods), but not necessarily.
     \item \textbf{Results.} What were the main results? You don't have to
     go too much into detail, i.e., report statistical test results. Explain
     what was tested which hypotheses were rejected and for which there was no proof to reject them.    
     \item \textbf{Conclusions.} Discuss the results within the broader context of
     the literature (you can either do it based on the paper or you can interpret
     the results based on your knowledge). What is the take-home message?
    \end{itemize}
\end{frame}
    
\section[Things to remember]{Things to remember}
\begin{frame}{Things to remember}
    \begin{itemize}
     \item Half an hour is less time than you think. \alert{Make up to 20 slides}.
     \item Begin by explaining the structure of the presentation.
     \item \alert{Use as many pictures/schemes as possible}.
     \item Don't use fancy transitions unless there is a good reason for that.
     \item Use colors wisely.
     \item Don't put too much bibliography in the slides. Unless it is something crucial.
     \item Rehearse your presentation before you give it.
    \end{itemize}
\end{frame}
    
\section[Formal requirements]{Formal requirements}
\begin{frame}
        \frametitle{Formal requirements}
        \only<1>{
            \begin{itemize}
            \item The presentation must be based on a \alert{scientific research paper}.
            \item The paper should use computational methods.
            \item The presentation should not exceed \alert{30 minutes}.
            \item People asking questions should prepare at least two questions based on 
            the paper. 
            \end{itemize}
        }
        \only<2>{
            \begin{block}{}
                Last but not least, the presentation must look aesthetically. Please, spend
                some time on formatting, checking spelling and grammar.
            \end{block}
        }
    \end{frame}


\end{document}