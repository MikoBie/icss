\documentclass[aspectratio=169]{beamer}
\usepackage{ulem}
\usepackage{tikz}
\usepackage{booktabs}
 \usepackage{graphicx,threeparttable,caption}
\usetikzlibrary{shapes,snakes}
\usepackage[beamer,customcolors]{hf-tikz}
\usepackage{nicematrix}
\usepackage{xcolor}
\usepackage{makecell}
\usepackage{array}
\usepackage{csquotes}
\usepackage{csquotes}
\usepackage{minted}
\captionsetup{labelformat=empty,labelsep=none}

\graphicspath{ {./png/} }

\usetikzlibrary{
    arrows,
    arrows.meta,
    shapes,
    positioning,
    shadows,
    trees,
    calc
}

\tikzset{%
    >={Latex[width=2mm,length=2mm]},
    % Specifications for style of nodes:
    plain/.style = {},
    base/.style = {
        plain,
        rectangle, rounded corners, draw=black,
        minimum width=1cm, minimum height=1cm,
        text centered, font=\sffamily\tiny\bfseries,
        fill=white, align=center
    },
    app/.style = {base, ellipse},
    data/.style = {base, fill=gray!30},
    action/.style = {base, circle, fill=red!30},
    note/.style = {app, fill=yellow},
    hl/.style={
    set fill color=red!80!black!40,
    set border color=red!80!black
    }
}


\AtBeginSection[]{
  \begin{frame}
  \vfill
  \centering
  \begin{beamercolorbox}[sep=8pt,center,shadow=true,rounded=true]{title}
    \usebeamerfont{title}\insertsectionhead\par%
  \end{beamercolorbox}
  \vfill
  \end{frame}
}
%\usecolortheme[orchid]{structure}
\usetheme[hideothersubsections]{PaloAlto}
\makeatletter
\patchcmd{\csq@bquote@i}{{#6}}{{\emph{#6}}}{}{}
\makeatother
%\usecolortheme{orchid}
%\usefonttheme{professionalfonts}
\newcommand{\soutthick}[1]{%
   \textcolor{red}{
   \renewcommand{\ULthickness}{1pt}%
      \sout{#1}%
   \renewcommand{\ULthickness}{.4pt}% Resetting to ulem default
   }
}
\newcommand{\centered}[1]{\begin{tabular}{l} #1 \end{tabular}}
\setbeamertemplate{section in toc}[square]
\setbeamertemplate{subsection in toc}[square]
\setbeamertemplate{secion in sidebar}[shaded]
\setbeamertemplate{items}[square]
\setbeamercovered{transparent} 

\title[]{Introduction to Computational Social Science}
\subtitle{Digital data sources and where to find them}
\author[]{Mikołaj Biesaga\\ \small{\color{blue}{\href{mailto:m.biesaga@uw.edu.pl}{m.biesaga@uw.edu.pl}}}}
\institute{\includegraphics[width = 4 cm]{uw.png}}
\date{\today}
\begin{document}
\begin{frame}
   \titlepage
\end{frame}

\section[Data]{Data}

\subsection[Data Sources]{Data Sources}
\begin{frame}
    \setbeamercovered{transparent}
    \frametitle{Data Sources}
    \only<+>{
        \begin{figure}
            \includegraphics[scale=.35]{png/data_everywhere.png}
        \end{figure}}
    \only<+>{
        \begin{figure}
            \includegraphics[scale = .5]{png/tracker.jpg}
        \end{figure}
    }
    \only<+>{
        \begin{itemize}
            \item<3> webpages
            \item<3> social media
            \item<3> smart devices
            \item<3> digital behavioral data
            \item<3> mobile phone networks
            \item<3> goverment data
            \item <3>...
        \end{itemize}
        \action<3>{\alert{The fact that you can get the data does not mean you should.}}
    }
\end{frame}

\section{Webscraping}

\begin{frame}
    \frametitle{Webscraping}
    \only<+>{
        \begin{figure}
            \includegraphics[scale = .5]{png/webscraping.jpg}
        \end{figure}
    }
    \only<+>{
        \begin{figure}
            \includegraphics[scale = .22]{png/politico.png}
            \caption{from \textcolor{blue}{\href{https://www.politico.eu/article/jacques-chirac-mourn-the-man-not-his-politics/}{POLITICO Europe}}}
        \end{figure}
    }
    \only<+>{
        \begin{definition}
            \emph{Webscraping} is a process of (usually) automatic extraction of data from a website or multiple websites. In other words, it is a form of copying the data from a website into a local database or spreadsheet.
        \end{definition}
    }
\end{frame}

\subsection{HTML}

\begin{frame}
    \frametitle{HyperText Markup Language}
    \begin{definition}{}
        \emph{HyperText Markup Language} (HTML) is the standard markup language for
        documents designed to be displayed in a web browser. It defines the
        content and structure of web content. It is often assisted by
        technologies such as Cascading Style Sheets (CSS) and scripting
        languages such as JavaScript.
    \end{definition}
\end{frame}

\begin{frame}[fragile]
\frametitle{HyperText Markup Language}
\begin{minted}[fontsize=\footnotesize]{html}
<!DOCTYPE html>
<html>
    <head>
        <title>
            Justyna Kowalczyk fandom
        </title>
    </head>
    <body>
        <h1>Why Justyna Kowalczyk is the best?</h1>
        <p>
            Because she is just <b>the best</b> cross-country 
            skier in the history of the sport. You can learn
            more about her amazing achievements visiting
            her Wikipedia webpage:
            https://pl.wikipedia.org/wiki/Justyna_Kowalczyk.
        </p>
    </body>
</html>
\end{minted}
\end{frame}

\begin{frame}
    \frametitle{HyperText Markup Language}
    \only<+>{
        \framesubtitle{What are tags?}
        Tags are used to mark up the start of an HTML element and they are enclosed
        in \textbf{angle brackets}. The most important is the <html> tag. Inside
        this tag, between <html> and </html> all other elements live. In the 
        example from the previous slide we had the following tags:
        \begin{itemize}
            \item {<head></head>} -- element contains meta-information about the document
            \item {<title></title>} -- element specifies a title for the document
            \item {<body></body>} -- element contains the visible page content
            \item {<h1></h1>} -- element defines a large heading
            \item {<p></p>} -- element define a paragraph
            \item {<b></b>} -- element define a boldface
        \end{itemize}
    }
\end{frame}

\section{API}

\begin{frame}
    \frametitle{Application Programming Interface}
    \only<+>{
        \begin{figure}
            \includegraphics[scale = .4]{png/api.jpg}
        \end{figure}
    }
    \only<+>{
        \begin{figure}
            \includegraphics[width = \textwidth]{png/twitter.png}
            \caption{from \textcolor{blue}{\href{https://developer.twitter.com/en/use-cases/analyze}{Developer X}}}
        \end{figure}
    }
    \only<+>{
        \begin{definition}
            \emph{Aplication Programming Interface} is a communication protocol between a client and a server intended to simplify the building of client-side software. In other words, it is a contract between the client and the server which defines the format of possible requests and the format of the response (i.e. format of the data).
        \end{definition}
    }
\end{frame}

\section{Examples}
\begin{frame}
    \frametitle{Reddit}
    \only<+>{
        \begin{figure}
            \centering
            \includegraphics[width = .3\textwidth]{reddit_log.png}
        \end{figure}
    }
    \only<+>{
        \begin{figure}
            \centering
            \includegraphics[width = \textwidth]{hicss.png}
        \end{figure}
    }
    \only<+>{
        \framesubtitle{(Roszczyńska-Kurasińska et al., 2025)}
        \begin{figure}
            \centering
            \includegraphics[width = .7\textwidth]{hicss-figure.png}
        \end{figure}

    }
    \frametitle{Wikipedia}
    \only<+>{
        \begin{figure}
            \centering
            \includegraphics[width = .3\textwidth]{wiki_logo.png}
        \end{figure}
    }
    \only<+>{
        \begin{figure}
            \centering
            \includegraphics[width = \textwidth]{wiki1.png}
        \end{figure}
    }
    \only<+>{
        \framesubtitle{(Wagner et al., 2015)}
        \begin{itemize}
            \item \textbf{Coverage bias} determines differences between the
            number of notable women and men portrayed on Wikipedia.             
            \item \textbf{Visibility bias} reflects how many articles about men
            or women make it to the front page of Wikipedia.        
            \item \textbf{Structural bias} quantifies gender homophily/disassortativity,
            i.e. gender-specific tendencies to preferably link articles of
            notable people with the same or different gender. 
            \item \textbf{Lexical bias} reveals inequalities in the words used
            to describe notable men and women on Wikipedia.
        \end{itemize}
    }
    \only<+>{
        \framesubtitle{(Wagner et al., 2015)}
        \begin{itemize}
            \item \textbf{Coverage bias:} Men and women are covered equally well
            in all six Wikipedia language editions. 
            \item \textbf{Visibility bias:} No evidence for male-bias in the
            selection procedure of articles that are featured on the startpage
            of the English Wikipedia.
            \item \textbf{Structural bias:} Women on Wikipedia tend
            to be more linked to men than vice versa. 
            \item \textbf{Lexical bias:} Romantic relationships and
            family-related issues are much more frequently discussed on
            Wikipedia articles about women than men.
        \end{itemize}
    }
    \only<+>{
        \begin{figure}
            \centering
            \includegraphics[width = \textwidth]{wiki2.png}
        \end{figure}
    }
    \only<+>{
        \framesubtitle{(Ruprechter et al., 2020)}
        \begin{figure}
            \centering
            \includegraphics[width = .8\textwidth]{wikipedia_chart.png}
        \end{figure}
    }
\end{frame}

\begin{frame}
    \frametitle{HEART project}
    \only<1,2,3>{
        \framesubtitle{Study design}
        \begin{tikzpicture}
            \node [text width = 4cm] at (-5,0) {\includegraphics[width = \textwidth]{survey.png}};
            \node at (-5,-2) {\footnotesize\textsc{Before Survey}};
            \node<2-> [text width = 4cm] at (0,0) {\includegraphics[width = \textwidth]{regular_visits.png}};
            \node<2-> at (0,-2) {\footnotesize\textsc{Regular Visits}};
            \node<3> [text width = 4cm] at (5,0) {\includegraphics[width = \textwidth]{survey.png}};
            \node<3> at (5,-2) {\footnotesize\textsc{After Survey}};

        \end{tikzpicture}
    }
    \only<4>{
        \framesubtitle{Expected workflow}
        \centering
        \begin{tikzpicture}
            \node [text width = 1cm] at (0,0) {\includegraphics[width = \textwidth]{questionnaire.png}};
            \node at (0,-1) {\footnotesize\textsc{Survey}};
            \node [text width = 1cm] at (2,0) {\includegraphics[width = \textwidth]{database.png}};
            \node at (2,-1) {\footnotesize \textsc{DataBase}};
            \node [text width = 1cm] at (4,0) {\includegraphics[width = \textwidth]{spreadsheet.png}};
            \node at (4, -1) {\footnotesize \textsc{Spreadsheet}};
            \node [text width = 1cm] at (6,0) {\includegraphics[width = \textwidth]{analysis.png}};
            \node at (6,-1) {\footnotesize \textsc{Analysis}};
            \draw [->, -stealth, thick] (.7,0) -- (1.3,0);
            \draw [->, -stealth, thick] (2.7,0) -- (3.3,0);
            \draw [->, -stealth, thick] (4.7,0) -- (5.3,0);
        \end{tikzpicture}
    }
    \only<5,6>{
        \framesubtitle{Actual workflow}
        \centering
        \begin{tikzpicture}
            \node [text width = 1cm] at (0,0) {\includegraphics[width = \textwidth]{questionnaire.png}};
            \node at (0,-1) {\footnotesize\textsc{Survey}};
            \node [text width = 1cm] at (2.5,0) {\includegraphics[width = \textwidth]{database.png}};
            \node at (2.5,-1) {\footnotesize \textsc{DataBase}};
            \node [text width = 1cm] at (5,0) {\includegraphics[width = \textwidth]{cleaning.png}};
            \node at (5,-1) {\footnotesize \textsc{Preprocessing}};
            \node [text width = 1cm] at (7.5,0) {\includegraphics[width = \textwidth]{spreadsheet.png}};
            \node at (7.5, -1) {\footnotesize \textsc{Spreadsheet}};
            \node [text width = 1cm] at (10,0) {\includegraphics[width = \textwidth]{analysis.png}};
            \node at (10,-1) {\footnotesize \textsc{Analysis}};
            \draw [->, -stealth, thick] (.7,0) -- (1.8,0);
            \draw [->, -stealth, thick] (3.2,0) -- node [above] {API} (4.3,0);
            \draw [->, -stealth, thick] (5.7,0) -- (6.8,0);
            \draw [->, -stealth, thick] (8.2,0) -- (9.3,0);
            \draw<6> [color = orange] (5,0) circle (.8cm);
        \end{tikzpicture}
    }
\end{frame}
\section{Migrations 2.0}

\begin{frame}
    \frametitle{Measuring Migrations 2.0}
    \begin{figure}
        \includegraphics[width = \textwidth]{png/table.png}
        \caption{Tjaden, 2021}
    \end{figure}
\end{frame}

\begin{frame}
    \frametitle{For the next week}
    \begin{enumerate}
        \item Find a scientific article that collects data either through API or webscrapping (Google Scholar). It should be published in a peer-reviewed social science journal and have at least 10 citations.
        \item Read it.
        \item Submit the title, authors, year of publication, journal, and maximum 200 words takeaway message in Google Classroom. It should consist of at least 4 sentences: problem statement, research question/hypothesis, what they did, main finding.
        \item Deadline: \textbf{14th November 2025, 11:59 AM}.
    \end{enumerate}

\end{frame}

\end{document}
