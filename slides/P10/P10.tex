\documentclass[aspectratio=169]{beamer}
\usepackage{ulem}
\usepackage{tikz}
\usepackage{booktabs}
\usepackage{graphicx,threeparttable,caption}
\usetikzlibrary{shapes,snakes}
\usepackage[beamer,customcolors]{hf-tikz}
\usepackage{nicematrix}
\usepackage{xcolor}
\usepackage{makecell}
\usepackage{array}
\usepackage{csquotes}
\usepackage{csquotes}
\usepackage{minted}
\captionsetup{labelformat=empty,labelsep=none}

\graphicspath{ {./png/} }

\usetikzlibrary{
    arrows,
    arrows.meta,
    shapes,
    positioning,
    shadows,
    trees,
    calc
}

\tikzset{%
    >={Latex[width=2mm,length=2mm]},
    % Specifications for style of nodes:
    plain/.style = {},
    base/.style = {
        plain,
        rectangle, rounded corners, draw=black,
        minimum width=1cm, minimum height=1cm,
        text centered, font=\sffamily\tiny\bfseries,
        fill=white, align=center
    },
    app/.style = {base, ellipse},
    data/.style = {base, fill=gray!30},
    action/.style = {base, circle, fill=red!30},
    note/.style = {app, fill=yellow},
    hl/.style={
    set fill color=red!80!black!40,
    set border color=red!80!black
    }
}


\AtBeginSection[]{
  \begin{frame}
  \vfill
  \centering
  \begin{beamercolorbox}[sep=8pt,center,shadow=true,rounded=true]{title}
    \usebeamerfont{title}\insertsectionhead\par%
  \end{beamercolorbox}
  \vfill
  \end{frame}
}
\setbeamercolor{alerted text}{fg=red}
%\usecolortheme[orchid]{structure}
\usetheme[hideothersubsections]{PaloAlto}
\makeatletter
\patchcmd{\csq@bquote@i}{{#6}}{{\emph{#6}}}{}{}
\makeatother
%\usecolortheme{orchid}
%\usefonttheme{professionalfonts}
\newcommand{\soutthick}[1]{%
   \textcolor{red}{
   \renewcommand{\ULthickness}{1pt}%
      \sout{#1}%
   \renewcommand{\ULthickness}{.4pt}% Resetting to ulem default
   }
}
\newcommand{\centered}[1]{\begin{tabular}{l} #1 \end{tabular}}
\setbeamertemplate{section in toc}[square]
\setbeamertemplate{subsection in toc}[square]
\setbeamertemplate{section in sidebar}[shaded]
\setbeamertemplate{items}[square]
\setbeamercovered{transparent} 

\title[]{Introduction to Computational Social Science}
\subtitle{Networks -- how small is the world?}
\author[]{Mikołaj Biesaga\\ \small{\color{blue}{\href{mailto:m.biesaga@uw.edu.pl}{m.biesaga@uw.edu.pl}}}}
\institute{\includegraphics[width = 4 cm]{uw.png}}
\date{\today}
\begin{document}
\begin{frame}
   \titlepage
\end{frame}

\begin{frame}
   \frametitle{Before we start}
   \begin{figure}
      \centering
      \includegraphics[width = .5\textwidth]{network.png}
      \caption{from \href{https://interactioninstitute.org/network-analysis-for-change-collaborations-clusters-champions-and-coach-weavers/}{\textcolor{blue}{Ogden, 2020}}}
   \end{figure}

\end{frame}

\section[The small world problem]{The small world problem}
\begin{frame}
   \frametitle{The small world problem}
   \framesubtitle{Milgram, 1967}
   \only<1>{
      There are two general philosophical views of the small world problem:
      \begin{minipage}{.45\textwidth}
         \begin{block}{}
            \alert{Any two people in the world can be linked}, no matter how remote from each other, can be linked in terms of intermediate acquaintances, and that the number of such intermediate links is relatively small.
         \end{block}
      \end{minipage}
      \begin{minipage}{.08\textwidth}
         \hfill
      \end{minipage}
      \begin{minipage}{.45\textwidth}
         \begin{block}{}
            There are unbridgeable gaps between various groups and that therefore, \alert{given any two people in the world, they will never link up} becasue people have circles of acuaintances which do not necessarily intersect.
         \end{block}
      \end{minipage}
   }
   \only<2>{
      \centering{
         \includegraphics[width = .9\textwidth]{milgram.png}
      }
   }
   \only<3>{
      \begin{minipage}{.49\textwidth}
         \includegraphics[width = \textwidth]{migram_graph.png}
      \end{minipage}
      \begin{minipage}{.49\textwidth}
         \begin{itemize}
            \item The median of intermediate persons was 5.
            \item Participants were three times more likely to send the package to a person of the same sex as to someone of the different sex.
         \end{itemize}
         \centering
         \includegraphics[width = .6\textwidth]{milgram_table.png}
      \end{minipage}
   }
\end{frame}

\begin{frame}
   \only<1>{
      \frametitle{Could it be a small world after all?}
      \framesubtitle{Kleinfeld, 2002}
      \begin{itemize}
               \item Only \alert{small} percentage of packages reached the final destination (Kansas -- 5\%; Nebraska -- 30\%).
               \item Not a representative sample -- the recruitment process favored sociable people.
               \item Social class separation -- low-income senders were only able to reach low-income recipients.
      \end{itemize}
   }
   \only<2>{
      \frametitle{How small is the world, really?}
      \framesubtitle{Watts, 2016}
      \begin{itemize}
         \item Facebook degrees of separation -- 4.57 (Edunov et al., 2016)
         \item Algorithmic vs. topological approach to the small world problem (Watts, 2016)
      \end{itemize}
      \begin{minipage}{.45\textwidth}
      \begin{block}{Topological approach}
         The small world problem is a property of the network. The degree of 
         separation refers to the underling structure of the network.
      \end{block}
      \end{minipage}
      \begin{minipage}{.08\textwidth}
         \hfill
      \end{minipage}
      \begin{minipage}{.45\textwidth}
      \begin{block}{Algorithmic approach}
         The small world problem is a property of the algorithm. The degree of separation refers to the ability to find the shortest path between two nodes.
      \end{block}
      \end{minipage}
   }

\end{frame}

\begin{frame}
   \frametitle{Six degrees of separation}
   \centering
   \includegraphics[width = .5\textwidth]{png/bacon.png}
\end{frame}

\section[Homophily]{Homophily}

\begin{frame}
   \frametitle{The small world network}
   \only<1>{
      \begin{figure}
         \centering
         \includegraphics[width = .5\textwidth]{network.png}
         \caption{from \href{https://interactioninstitute.org/network-analysis-for-change-collaborations-clusters-champions-and-coach-weavers/}{\textcolor{blue}{Ogden, 2020}}}
      \end{figure}
   }
   \only<2>{
      \framesubtitle{Homophily}
      \begin{block}{Homophily}
         \underline{Homophily} is a concept in sociology describing the tendency of individuals to associate and bond with similar others, as in the proverb "birds of a feather flock together".
      \end{block}
   }
   \only<3>{
      \framesubtitle{Homophily}
      \begin{figure}
         \centering
         \includegraphics[width = .7\textwidth]{homophily.png}
         \caption{Lee et al. (2019)}
      \end{figure}
   }
   \only<4>{
      \framesubtitle{Lee et al. (2019)}
      \begin{itemize}
         \item When homophily is \alert{high}, both minority and majority groups
         tend to \alert{overestimate} their own size, whereas when homophily is
         \alert{low}, both groups tend to \alert{underestimate} their own size.
         \item The \alert{smaller the size of the minority group}, the more
         its size is \alert{overestimated} by both minority and majority groups.
      \end{itemize}
   }
\end{frame}

\section[Weak ties]{Weak ties}
\begin{frame}
   \frametitle{The power of weak ties}
   \only<1>{
      \framesubtitle{Granovetter, 1973}
      \begin{figure}
         \centering
         \includegraphics[width = .5\textwidth]{network.png}
         \caption{from \href{https://interactioninstitute.org/network-analysis-for-change-collaborations-clusters-champions-and-coach-weavers/}{\textcolor{blue}{Ogden, 2020}}}
      \end{figure}
   }
   \only<2>{
      \begin{block}{}
         Granovetter (1973) identified weak ties as a key source of “diffusion of
         influence and information, mobility opportunity, and community
         organization.”
      \end{block}
   }
   \only<3,5>{
      \framesubtitle{Rajkumar et al., 2022}
      \begin{itemize}
         \item A five-year experiment involving LinkedIn’s “People You May Know”
         (PYMK) algorithm, which suggests new connections to site users. 
         \item A weak tie was defined two fold:
         \begin{itemize}
            \item message intensity
            \item number of mutual friends
         \end{itemize}
         \item The experiment involved around \alert{20 million} LinkedIn users, who
         over the five years ended up creating about \alert{2 billion new connections}
         on the site, recorded over \alert{70 million job applications}, and wound up
         accepting \alert{600,000 new jobs} identified through the site.
      \end{itemize}

   }
   \only<4>{
      \begin{figure}
         \centering
         \includegraphics[width = .8\textwidth]{weak_ties.png}
         \caption{Rajkumar et al., 2022}
      \end{figure}
   }
   \only<6>{
      \begin{figure}
         \begin{minipage}{.49\textwidth}
            \includegraphics[width = \textwidth]{interaction_intensity.png}
         \end{minipage}
         \begin{minipage}{.49\textwidth}
            \includegraphics[width = \textwidth]{number_friends.png}
         \end{minipage}
         \caption{Rajkumar et al., 2022}
      \end{figure}

   }
\end{frame}

%% \section[Migration Studies]{Migration Studies}
%% \begin{frame}
%%    \frametitle{Migration Studies -- network approach}
%%    \framesubtitle{Massey et al., 1993}
%%    \begin{block}{}
%%       Migrant networks are sets of interpersonal ties that connect migrants,
%%       former migrants, and nonmigrants in origin and destination areas through ties
%%       of kinship, friendship, and shared community origin.
%%    \end{block}
%% \end{frame}

\begin{frame}
   \frametitle{Bibliography}
   \footnotesize
   \begin{enumerate}
      \item Edunov, S., Bhagat, S., Burke, M., Diuk, C., \& Filiz, I. O., (2016). \href{https://research.facebook.com/blog/2016/2/three-and-a-half-degrees-of-separation/}{\textcolor{blue}{Three and a half degrees of separation}}
      \item Granovetter, M. S. (1973). The Strength of Weak Ties. American
      Journal of Sociology, 78(6), 1360–1380.
      \item Kleinfeld, J. S. (2002). The small world problem. 39, 61–66.
      \item Lee, E., Karimi, F., Wagner, C., Jo, H.-H., Strohmaier, M., \&
      Galesic, M. (2019). Homophily and minority-group size explain perception
      biases in social networks. Nature Human Behaviour, 3(10), 1078–1087.
      \href{https://doi.org/10.1038/s41562-019-0677-4}{\textcolor{blue}{https://doi.org/10.1038/s41562-019-0677-4}}
      %% \item Massey, D. S., Arango, J., Hugo, G., Kouaouci, A., Pellegrino, A.,
      %% \& Taylor, J. E. (1993). Theories of International Migration: A Review and
      %% Appraisal. Population and Development Review, 19(3), 431.
      %% \href{https://doi.org/10.2307/2938462}{\textcolor{blue}{https://doi.org/10.2307/2938462}}
      \item Milgram, S. (1967). The small-world problem. Psychology Today, 1(1), 61–67.
      \item Rajkumar, K., Saint-Jacques, G., Bojinov, I., Brynjolfsson, E., \&
      Aral, S. (2022). A causal test of the strength of weak ties. Science,
      377(6612), 1304–1310. \href{https://doi.org/10.1126/science.abl4476}{\textcolor{blue}{https://doi.org/10.1126/science.abl4476}}
      \item Watts, D. (2016). \href{https://medium.com/@duncanjwatts/how-small-is-the-world-really-736fa21808ba}{\textcolor{blue}{How small is the world, really?}}
   \end{enumerate}
\end{frame}

\begin{frame}
    \frametitle{For the class next week}
    \begin{enumerate}
        \item Find a scientific article that uses the network approach (Google Scholar). It should be published in a peer-reviewed social science journal and have at least 10 citations.
        \item Read it.
        \item Submit the title, authors, year of publication, journal, and maximum 200 words takeaway message in Google Classroom. It should consist of at least 4 sentences: problem statement, research question/hypothesis, what they did, main finding.
        \item Deadline: \textbf{16th January 2026, 11:59 AM}.
    \end{enumerate}
\end{frame}

\end{document}