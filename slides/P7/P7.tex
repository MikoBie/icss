\documentclass[aspectratio=169]{beamer}
\usepackage{ulem}
\usepackage{tikz}
\usepackage{booktabs}
\usepackage{graphicx,threeparttable,caption}
\usetikzlibrary{shapes,snakes}
\usepackage[beamer,customcolors]{hf-tikz}
\usepackage{nicematrix}
\usepackage{xcolor}
\usepackage{makecell}
\usepackage{array}
\usepackage{csquotes}
\usepackage{csquotes}
\usepackage{minted}
\usepackage{animate}
\captionsetup{labelformat=empty,labelsep=none}

\graphicspath{ {./png/} }

\usetikzlibrary{
    arrows,
    arrows.meta,
    shapes,
    positioning,
    shadows,
    trees,
    calc
}

\tikzset{%
    >={Latex[width=2mm,length=2mm]},
    % Specifications for style of nodes:
    plain/.style = {},
    base/.style = {
        plain,
        rectangle, rounded corners, draw=black,
        minimum width=1cm, minimum height=1cm,
        text centered, font=\sffamily\tiny\bfseries,
        fill=white, align=center
    },
    app/.style = {base, ellipse},
    data/.style = {base, fill=gray!30},
    action/.style = {base, circle, fill=red!30},
    note/.style = {app, fill=yellow},
    hl/.style={
    set fill color=red!80!black!40,
    set border color=red!80!black
    }
}


\AtBeginSection[]{
  \begin{frame}
  \vfill
  \centering
  \begin{beamercolorbox}[sep=8pt,center,shadow=true,rounded=true]{title}
    \usebeamerfont{title}\insertsectionhead\par%
  \end{beamercolorbox}
  \vfill
  \end{frame}
}
\setbeamercolor{alerted text}{fg=red}
%\usecolortheme[orchid]{structure}
\usetheme[hideothersubsections]{PaloAlto}
\makeatletter
\patchcmd{\csq@bquote@i}{{#6}}{{\emph{#6}}}{}{}
\makeatother
%\usecolortheme{orchid}
%\usefonttheme{professionalfonts}
\newcommand{\soutthick}[1]{%
   \textcolor{red}{
   \renewcommand{\ULthickness}{1pt}%
      \sout{#1}%
   \renewcommand{\ULthickness}{.4pt}% Resetting to ulem default
   }
}
\definecolor{YELLOW}{RGB}{255, 234, 33}
\definecolor{BLUE}{RGB}{70, 174, 194}
\definecolor{RED}{RGB}{199, 36, 106}
\newcommand{\centered}[1]{\begin{tabular}{l} #1 \end{tabular}}
\setbeamertemplate{section in toc}[square]
\setbeamertemplate{subsection in toc}[square]
\setbeamertemplate{section in sidebar}[shaded]
\setbeamertemplate{items}[square]
\setbeamercovered{transparent} 

\title[]{Introduction to Computational Social Science}
\subtitle{Text as data -- discussion}
\author[]{Mikołaj Biesaga\\ \small{\color{blue}{\href{mailto:m.biesaga@uw.edu.pl}{m.biesaga@uw.edu.pl}}}}
\institute{\includegraphics[width = 4 cm]{uw.png}}
\date{\today}
\begin{document}
\begin{frame}
   \titlepage
\end{frame}

\begin{frame}
   \frametitle{Narratives}
   \begin{itemize}
      \item<1-> Humans are "storytelling animals" (Gottschall, 2012). People are not computers. We think not with data, but with stories (Robinson \& Hawpe, 1986).
      \item<2-> The Psychology of Life Stories (McAdams, 2001).
      \item<3-> Whenever we find ourselves in a new situation or face a decision, we use narratives to understand the circumstances and compare potential scenarios (Beach, 2021). Stories are thus the primary tool by which we experience the world. They provide meaning to reality (Popova, 2015). 
      \item<4-> Many studies suggest that fake news tends to generate more attention and spread faster than true information. One of the reasons for that may be that they represent attractive narrative structures (Baptista \& Gradim, 2020). \alert{Fake news is perceived as more interesting and more relevant.}
   \end{itemize}
\end{frame}

\begin{frame}
   \frametitle{What are narratives?}
   \only<1>{
      Narratives tell us:
      \begin{enumerate}
         \item What is this? 
         \item How is the situation changing, and what should we do about it?
         \item Who am I in this situation with? 
      \end{enumerate}
      Thus, narratives perform 3 main functions:
      \begin{enumerate}
         \item Organising our perception
         \item Guiding our actions
         \item Synchronising collective efforts
      \end{enumerate}
   }
   \only<2>{
      \includegraphics[width = \textwidth]{narratives_types.png}
   }
   \only<3>{
      \begin{definition}
         Narratives are \alert{NOT} “themes”. What is said is only one element of
         the narrative. Equally important is the overall structure of change and
         the sense of community (who are we working with? against whom?). Seemingly
         different stories about vaccines, masks, and lockdowns may turn out to be the
         same story. Only the superficial props are changing. And the change is
         irrelevant because it has never really been about vaccines or masks, but
         about a deeper problem symbolized by these preventive measures.
      \end{definition}
   }
\end{frame}

\begin{frame}
   \frametitle{Methodology}

   \centering

   \begin{tikzpicture}[node distance=3cm]
      % Specification of nodes (position, etc.)
      \node (topic)   [base, fill = YELLOW] {Topic Modeling};
      \node (extraction) [base, right of=topic, fill = BLUE] {Narratives Extraction};
      \node (annotation) [base, right of=extraction, fill = BLUE] {Data Annotation};
      \node (analysis) [base, right of=annotation, fill = RED] {Quantitative Analysis};
      % Specification of paths
      \draw[->] (topic) -- (extraction);
      \draw[->] (extraction) -- (annotation);
      \draw[->] (annotation) -- (analysis);
   \end{tikzpicture}

\end{frame}

\begin{frame}
   \frametitle{Identified Narratives}
   \only<1>{
      \centering
      \includegraphics[width = .7\textwidth]{climate_narratives.png}
   }
   \only<2>{
      \centering
      \includegraphics[width = \textwidth]{narrative_table.png}
   }
   \only<3>{
      \framesubtitle{We need to resist the green dystopia}
      \begin{minipage}{.49\textwidth}
         \includegraphics[width = \textwidth]{N1.png}
      \end{minipage}
      \begin{minipage}{.49\textwidth}
         \includegraphics[width = \textwidth]{N1_1.png}
      \end{minipage}
   }
   \only<4>{
      \framesubtitle{Let's not panic!}
      \begin{minipage}{.49\textwidth}
         \includegraphics[width = \textwidth]{N3.jpg}
      \end{minipage}
      \begin{minipage}{.49\textwidth}
         \includegraphics[width = \textwidth]{N3_1.png}
      \end{minipage}
   }
   \only<5>{
      \framesubtitle{Distribution of narratives}
      \centering
      \includegraphics[width = .4\textwidth]{narratives-dists.png}

   }
\end{frame}

\begin{frame}
   \frametitle{Results}
   \framesubtitle{Narrative Stances}
   \only<1>{
      \centering
      \includegraphics[width = .58\textwidth]{triangle.png}
   }
   \only<2>{
      \centering
      \includegraphics[width = .5\textwidth]{distribution_stances.png}
   }
   \only<3>{
      \centering
      \includegraphics[width = .9\textwidth]{ts_narratives_PL.png}
      \includegraphics[width = .9\textwidth]{ts_narratives_ESP.png}
   }
   \only<4>{
      \centering
      \includegraphics[width = .9\textwidth]{ts_narratives_UK.png}
      \includegraphics[width = .9\textwidth]{ts_narratives_FR.png}
   }
\end{frame}

\begin{frame}
   \frametitle{Next steps}
   \begin{enumerate}
      \item Use a machine learning tool for annotating more data.
      \item Use Large Language Models to scale up the narrative approach.
   \end{enumerate}
\end{frame}

\begin{frame}
   \frametitle{Discussion}
   \begin{enumerate}
      \item Imagine you want to investigate what the opinions are on climate
      change? How would you approach it? What sources would you use?
      \item What are the possible pitfalls and risks of NLP tools?
      \item What topic modeling, sentiment analysis, automatic classification
      can offer to psychologists?
   \end{enumerate}
\end{frame}
\end{document}