\documentclass[aspectratio=169]{beamer}
\usepackage{ulem}
\usepackage{tikz}
\usepackage{booktabs}
\usepackage{graphicx,threeparttable,caption}
\usetikzlibrary{shapes,snakes}
\usepackage[beamer,customcolors]{hf-tikz}
\usepackage{nicematrix}
\usepackage{xcolor}
\usepackage{makecell}
\usepackage{array}
\usepackage{csquotes}
\usepackage{csquotes}
\usepackage{minted}
\usepackage{animate}
\captionsetup{labelformat=empty,labelsep=none}

\graphicspath{ {./png/} }

\usetikzlibrary{
    arrows,
    arrows.meta,
    shapes,
    positioning,
    shadows,
    trees,
    calc
}

\tikzset{%
    >={Latex[width=2mm,length=2mm]},
    % Specifications for style of nodes:
    plain/.style = {},
    base/.style = {
        plain,
        rectangle, rounded corners, draw=black,
        minimum width=1cm, minimum height=1cm,
        text centered, font=\sffamily\tiny\bfseries,
        fill=white, align=center
    },
    app/.style = {base, ellipse},
    data/.style = {base, fill=gray!30},
    action/.style = {base, circle, fill=red!30},
    note/.style = {app, fill=yellow},
    hl/.style={
    set fill color=red!80!black!40,
    set border color=red!80!black
    }
}


\AtBeginSection[]{
  \begin{frame}
  \vfill
  \centering
  \begin{beamercolorbox}[sep=8pt,center,shadow=true,rounded=true]{title}
    \usebeamerfont{title}\insertsectionhead\par%
  \end{beamercolorbox}
  \vfill
  \end{frame}
}
\setbeamercolor{alerted text}{fg=red}
%\usecolortheme[orchid]{structure}
\usetheme[hideothersubsections]{PaloAlto}
\makeatletter
\patchcmd{\csq@bquote@i}{{#6}}{{\emph{#6}}}{}{}
\makeatother
%\usecolortheme{orchid}
%\usefonttheme{professionalfonts}
\newcommand{\soutthick}[1]{%
   \textcolor{red}{
   \renewcommand{\ULthickness}{1pt}%
      \sout{#1}%
   \renewcommand{\ULthickness}{.4pt}% Resetting to ulem default
   }
}
\newcommand{\centered}[1]{\begin{tabular}{l} #1 \end{tabular}}
\setbeamertemplate{section in toc}[square]
\setbeamertemplate{subsection in toc}[square]
\setbeamertemplate{section in sidebar}[shaded]
\setbeamertemplate{items}[square]
\setbeamercovered{transparent} 

\title[]{Introduction to Computational Social Science}
\subtitle{Computer simulation -- why simulate?}
\author[]{Mikołaj Biesaga\\ \small{\color{blue}{\href{mailto:m.biesaga@uw.edu.pl}{m.biesaga@uw.edu.pl}}}}
\institute{\includegraphics[width = 4 cm]{uw.png}}
\date{\today}
\begin{document}
\begin{frame}
   \titlepage
\end{frame}

\part[Projects]{Projects}
\frame{\partpage}
\begin{frame}{Possible forms of the project}
    \begin{enumerate}
     \item \textbf{Research project proposal.} A short proposal for a research
     project using some of the methods we discussed in the class. Details are
     discussed in \textit{Structure of the proposal} section.
     \item \textbf{Simple research project and report.} Simple data-driven
     project in which you gather some data from an external web-based source
     (i.e. from an API or through webscraping), process it and use it to answer
     a simple research question. Details are discussed in \textit{Structure of
     the project} section.
    \end{enumerate}
\end{frame}
    
\section[Structure of the project]{Structure of the project}
\begin{frame}{Structure of the proposal | general information}
    \begin{itemize}
     \item \textbf{Introduction and problem statement.} This section should
     explain what is the project about in general and why the given problem
     matters. The significance of the problem may be justified both in terms of
     its theoretical/scientific or societal/business relevance.
     \item \textbf{Research question.} What is the specific research question
     you will try to answer in the project? It can be either formulated as a
     strict hypothesis or as an exploratory question (i.e. not assuming any
     particular effect or mechanism).
     \item \textbf{Research methods.} This section should discuss, at a general
     level, what data do you plan to use and how will analyze it. In particular,
     you should clearly explain why the data and methods you chose will allow
     you to answer your research question.
    \end{itemize}
\end{frame}
    
\begin{frame}{Structure of the proposal | research methods}
    \begin{itemize}
     \item \textbf{Data collection.} What is the data you are planning to use?
     Where is it stored and how do you plan to obtain/extract it?  Are there any
     possible obstacles and if there are what can be done to minimize the risk
     of failure?
     \item \textbf{Data preprocessing and storage.} How will you preprocess data
     and in what format will store it for later use? For instance, when scraping
     you may decide to data gathered from every individual website to a JSON
     object and store all data in \texttt{.jsonl} file (JSON lines).
     \item \textbf{Analytic methods.} What analytic methods will you apply to
     your data in order to answer your question? This may include some methods
     that we discussed in the class (i.e. sentiment analysis or some other
     natural language processing methods), but you should also use your general
     statistical knowledge to decide how you want to model your final data.
    \end{itemize}
\end{frame}
    
\begin{frame}{Structure of the project | general methods}
    \begin{itemize}
     \item The project should consist of two main items:
     \begin{enumerate}
     \item \textbf{Notebook} with code and description of the project.
     More on that below.
     \item File with the \textbf{raw data} as was extracted from the external
     data source (e.g. API or webpage).
     \end{enumerate}
     \item The notebook should include a short introduction describing the
     problem and why it is significant.
     \item The specific research question should be introduced.
     \item Data source should be discussed. In particular in relation to the
     research question.
     \item Data analysis methods should be discussed, also in relation to the
     research question.
     \item Code and text should be mixed in the notebook in a way that
     facilitates understanding of the adopted research methods and obtained
     results.
    \end{itemize}
\end{frame}
    
\section[Things to remember]{Things to remember}
\begin{frame}{Things to remember}
    \begin{itemize}
     \item Make sure you will be able to gather data legally.  For instance, if
     you want to scrape some websites you should try to determine whether they
     allow scraping. You can check this by looking at the so-called
     \texttt{robots.txt} of a website.  It can be found at
     \texttt{<url>/robots.txt}. For instance, you can see Facebook configuration
     at \texttt{facebook.com/robots.txt}.
     \item Be kind for your data source. If there is a platform that exposes an
     API you should use it instead of scraping data directly from its webpage.
     \item Think about proper format for storing data. If you plan to use your
     data to compute a statistical model in SPSS may be JSON lines are not the
     best and you should consider storing your data as a simple CSV file?
    \end{itemize}
\end{frame}
    
\section[Formal requirements]{Formal requirements}
\begin{frame}
        \frametitle{Formal requirements}
        \only<1>{
            \begin{itemize}
            \item Regardless whether you chose a project or a proposal the
            deadline for submitting it is on \textbf{XX.XX.2026 23:59:59}.
            \item If you decide to submit a project please send us two files:
            the notebook and the raw data file. Please, remember to rename them
            properly FP\_<SURNAME>\_<NAME>.ipynb and FP\_<SURNAME>\_<NAME>.jl
            (the data file might have the format of your choice, it doesn't have
            to be a JSON line file).
            \item If you decide to submit a proposal your work should not be
            shorter than 3 pages and not longer than 5 (interline: single, font
            size: 12). Please, remember to rename your file properly
            FP\_<SURNAME>\_<NAME>.docx.
            \item Regardless whether you will choose proposal or project you
            must include at least one bibliography item.
            \end{itemize}
        }
        \only<2>{
            \begin{block}{}
                Last but not least, it must look aesthetically. Please, spend
                some time on formatting and checking spelling and grammar.
            \end{block}
        }
    \end{frame}
\part[Computer Simulations]{Computer Simulations}
\frame{\partpage}
\section[Agent-based models]{Agent-based models}
\begin{frame}
   \frametitle{Agent-based models}
   \only<1>{
      \begin{figure}
         \centering
         \includegraphics[width = .4\textwidth]{agent_based_models.png}
         \caption{Galan et al., 2013}
      \end{figure}
   }
   \only<2>{
      \begin{block}{Agent-based models}
         Agent-based models are discrete-time computer simulations used to study
         the interactions between people, things, places, and time. They are
         stochastic models built from the bottom up meaning individual agents
         (often people in epidemiology) are assigned certain attributes.
         \alert{The agents are programmed to behave and interact with other
         agents and the environment in certain ways.} These interactions produce
         emergent effects that may differ from effects of individual agents.
      \end{block}
   }
\end{frame}

\section[Models of segregation]{Models of segregation}
\begin{frame}
   \frametitle{Dynamic models of segregation}
   \framesubtitle{Schelling, 1971}
   \only<1>{
      \begin{itemize}
      \item In 1965, the last of \href{https://en.wikipedia.org/wiki/Jim_Crow_laws}{\textcolor{blue}{Jim Crow's}} racial segregation laws were overturned.
      \item Despite much effort and investment, segregation still remains a major issue in the U.S. and elsewhere to this date (Cassidy, 2016).
      \item "If an individual is surrounded by more individuals of different type than the number of individuals of own type, then the individual moves from the current location to a random empty location." (Schelling, 1971)
   \end{itemize}
   }
   \only<2,3>{
      \begin{figure}
      \centering
      \includegraphics<2,3>[width = \textwidth]{schelling1.png}
      \includegraphics<3>[width = \textwidth]{schelling2.png}
      \caption{Schelling, 1971}
      \end{figure}
   }

\end{frame}

\begin{frame}
   \frametitle{The world of polygons}
   \begin{figure}
      \includegraphics[width = .5\textwidth]{polygons.png}
      \caption{by Nicky Case from \href{https://ncase.me/polygons/}{\textcolor{blue}{https://ncase.me/polygons/}}}
   \end{figure}

\end{frame}

\begin{frame}
   \frametitle{\large Modeling Dynamics of Multicultural Integration and Conflict}
   \framesubtitle{de Raad, Nowak, \& Borkowski, 2013}
   \only<1,2>{
      \only<1>{
         \begin{itemize}
            \item Acculturation vs Segregation.
            \item A positive attitude means that contact with that person increases happiness, a negative attitude is related to a decrease.
            \item Although happiness increases with the number of positively valued contacts, this increase is not linear.
            \item Individuals, or agents, were assigned to a group, and given an attitude towards the own and other group ranging from $-1$ to $+1$.
            \item A decision to move would be made if a spot would provide a higher level of happiness than the current location.
         \end{itemize}
      }
      \only<2>{
         $$H = A_{own} \times \sqrt{N_{own}} + A_{other} \times \sqrt{N_{other}}$$
      }
   }
   \only<3>{
      \begin{figure}
         \begin{minipage}{.39\textwidth}
            \includegraphics[width = \textwidth]{wouter1.png}
         \end{minipage}
         \begin{minipage}{.39\textwidth}
            \includegraphics[width = \textwidth]{wouter2.png}
         \end{minipage}
         \caption{\footnotesize\textbf{Left panel:} Initial state of a
         simulation in which agents of two groups – depicted in two shades of
         gray are randomly distributed on a grid.  Black color indicates empty
         spaces to which agents can potentially move. \textbf{Right panel:}
         Pattern of intergroup contact between agents with mutual negative
         attitudes.}
      \end{figure}
   }
   \only<4>{
      \begin{figure}
         \includegraphics[width = .8\textwidth]{wouter3.png}
         \caption{Pattern of intergroup contact between agents with equal and
         mutually positive attitudes towards each other. In the picture on the
         left mutual attitudes are 1.00; in the right picture the mutual
         attitudes are 0.50.}
      \end{figure}
   }
   \only<5>{
      \begin{figure}
         \includegraphics[width = .8\textwidth]{wouter4.png}
         \caption{\textbf{Left panel:} A pattern of intergroup contact between
         mutually tolerant agents; attitude towards the other group is 0.00.
         \textbf{Right panel:} the lighter shaded group has a moderately
         positive attitude of 0.50, the dark shaded group has a neutral attitude
         of 0.00.}
      \end{figure}
   }

\end{frame}
\begin{frame}
   \frametitle{Bibliography}
   \scriptsize
   \begin{enumerate}
      \item Galán, J. M., Izquierdo, L. R., Izquierdo, S. S., Santos, J. I., Del
      Olmo, R., \& López-Paredes, A. (2013). Checking Simulations: Detecting and
      Avoiding Errors and Artefacts. In B. Edmonds \& R. Meyer (Eds.),
      Simulating Social Complexity (pp. 95–116). Springer Berlin Heidelberg.
      \item De Raad, W. E., Nowak, A., \& Borkowski, W. (2013). Modeling
      Dynamics of Multicultural Integration and Conflict. In K. Sycara, M.
      Gelfand, \& A. Abbe (Eds.), Models for Intercultural Collaboration and
      Negotiation (Vol. 6, pp. 183–197). Springer Netherlands.
      \item Schelling, T. C. (1971). DYNAMIC MODELS OF SEGREGATION. Journal of
      Mathematical Psychology, 1, 143–186.
   \end{enumerate}
\end{frame}

\end{document}